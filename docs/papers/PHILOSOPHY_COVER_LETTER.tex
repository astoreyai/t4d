\documentclass[11pt]{letter}
\usepackage[margin=1in]{geometry}
\usepackage{times}

\signature{Aaron W. Storey\\
Department of Computer Science\\
Clarkson University\\
Potsdam, NY 13699\\
\texttt{storeyaw@clarkson.edu}\\
ORCID: 0009-0009-5560-0015}

\address{Aaron W. Storey\\
Department of Computer Science\\
Clarkson University\\
Potsdam, NY 13699}

\begin{document}
\begin{letter}{Editor-in-Chief\\
Philosophy \& Technology}

\opening{Dear Editor,}

I am pleased to submit the manuscript entitled ``\textbf{What Does It Mean for an AI to Remember? Epistemological and Metaphysical Foundations of Machine Memory}'' for consideration for publication in Philosophy \& Technology.

\textbf{Summary:} As AI systems increasingly incorporate persistent memory architectures, fundamental philosophical questions arise about the nature of machine memory. This paper examines whether computational memory systems can possess real memory or only simulate it, drawing on epistemology, philosophy of mind, and cognitive science.

\textbf{Key Contributions:}
\begin{enumerate}
    \item The \textbf{Continuity Criterion}: A framework for distinguishing genuine memory from sophisticated retrieval---a system has real memory if removing its memory state would change \textit{what it is}, not merely what it knows
    \item Analysis of philosophical deficits in current AI memory systems: lack of intentionality, absence of reconstruction, missing emotional modulation, and shallow grounding
    \item Application of extended mind thesis to distributed AI memory architectures, with engagement with Rupert's objections
    \item Ethical analysis of memory governance, differential memory, identity integrity, and adversarial memory attacks
\end{enumerate}

\textbf{Relevance to Philosophy \& Technology:} This paper bridges technical AI development with classical philosophical concerns about memory, knowledge, and identity. It engages with Locke, Brentano, Searle, Clark and Chalmers, and contemporary philosophy of mind to examine what building memory systems for AI requires us to think about the nature of memory itself.

\textbf{Declarations:}
\begin{itemize}
    \item No conflicts of interest to declare
    \item This manuscript has not been submitted elsewhere
    \item A companion technical paper has been submitted to IEEE Transactions on AI
\end{itemize}

Thank you for considering this submission. I believe this work addresses questions of growing importance as AI memory systems become more sophisticated.

\closing{Sincerely,}

\end{letter}
\end{document}
